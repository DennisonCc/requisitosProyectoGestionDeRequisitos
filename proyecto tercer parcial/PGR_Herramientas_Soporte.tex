\documentclass[12pt,a4paper]{article}
\usepackage[utf8]{inputenc}Para la selección de la herramienta más adecuada se establecieron los siguientes criterios de evaluación, priorizando opciones gratuitas y accesibles para proyectos académicos:

\begin{itemize}
    \item \textbf{Costo}: Accesibilidad económica - prioridad a herramientas gratuitas
    \item \textbf{Funcionalidad}: Capacidades específicas para gestión de requisitos
    \item \textbf{Usabilidad}: Facilidad de uso e interfaz intuitiva
    \item \textbf{Colaboración}: Soporte para trabajo en equipo
    \item \textbf{Integración}: Compatibilidad con otras her\textbf{Dashboard del Proyecto:}\subsection{Ventajas de G\begin{enumerate}
    \item \textbf{Curva de aprendizaje}: Requiere familiarización con GitHub si no se ha usado antes
    \item \textbf{Funcionalidades específicas}: Menos características especializadas para requisitos formales
    \item \textbf{Dependencia de internet}: Herramienta basada en la nube
    \item \textbf{Estructura}: Requiere disciplina en el uso de convenciones y templates
\end{enumerate}Projects para TutoESPEcialistas}

\begin{enumerate}
    \item \textbf{Completamente gratuito}: Sin restricciones para uso académico o profesional
    \item \textbf{Integración nativa}: Perfecto para proyectos de desarrollo de software
    \item \textbf{Familiaridad}: Ampliamente conocido en entornos académicos
    \item \textbf{Escalabilidad ilimitada}: Puede crecer sin costos adicionales
    \item \textbf{Colaboración efectiva}: Herramientas robustas para trabajo en equipo
    \item \textbf{Documentación integrada}: Wiki y README para documentación adicional
    \item \textbf{Control de versiones}: Integración natural con Git
    \item \textbf{Comunidad}: Acceso a la comunidad global de desarrolladores
\end{enumerate}mize}
    \item Número total de requisitos por estado (columnas en board)
    \item Distribución por prioridad (filtros por labels)
    \item Progreso por milestone
    \item Issues abiertas vs cerradas
    \item Actividad reciente del equipo
\end{itemize}

\textbf{Reportes Nativos de GitHub:}
\begin{itemize}
    \item Insights del repositorio con métricas de actividad
    \item Project insights con análisis de flujo de trabajo
    \item Búsquedas guardadas para requisitos específicos
    \item Notificaciones automáticas de cambios
    \item Integración con GitHub Pages para documentación
\end{itemize}tas
    \item \textbf{Escalabilidad}: Capacidad de crecer con el proyecto
    \item \textbf{Soporte}: Disponibilidad de documentación y ayuda
\end{itemize}e[spanish]{babel}
\usepackage{geometry}
\usepackage{array}
\usepackage{booktabs}
\usepackage{longtable}
\usepackage{multirow}
\usepackage{xcolor}
\usepackage{graphicx}
\usepackage{float}
\usepackage{hyperref}
\usepackage{fancyhdr}
\usepackage{titlesec}

% Configuración de página
\geometry{margin=2.5cm}
\pagestyle{fancy}
\fancyhf{}
\fancyhead[L]{Plan de Gestión de Requisitos - TutoESPEcialistas}
\fancyhead[R]{Grupo 4}
\fancyfoot[C]{\thepage}

% Configuración de títulos
\titleformat{\section}{\Large\bfseries}{\thesection}{1em}{}
\titleformat{\subsection}{\large\bfseries}{\thesubsection}{1em}{}
\titleformat{\subsubsection}{\normalsize\bfseries}{\thesubsubsection}{1em}{}

% Colores para las tablas
\definecolor{headercolor}{RGB}{51, 102, 153}
\definecolor{lightblue}{RGB}{220, 230, 241}
\definecolor{lightgray}{RGB}{245, 245, 245}

\title{\textbf{Plan de Gestión de Requisitos\\Sistema TutoESPEcialistas\\Herramientas de Soporte}}
\author{Grupo 4\\Chalacan Dennison\\Sandoval Fernando\\Grijalva Judá}
\date{Agosto 2025}

\begin{document}

\maketitle
\newpage

\tableofcontents
\newpage

\section{Herramientas de Soporte para la Gestión de Requisitos}

\subsection{Introducción}

La gestión efectiva de requisitos requiere el uso de herramientas especializadas que faciliten la documentación, seguimiento, trazabilidad y control de cambios de los requisitos a lo largo del ciclo de vida del proyecto. Para el sistema TutoESPEcialistas, se ha realizado un análisis comparativo de las principales herramientas disponibles en el mercado, considerando factores como funcionalidad, costo, facilidad de uso y compatibilidad con las necesidades específicas del proyecto.

\subsection{Definiciones y Conceptos Clave}

\subsubsection{Gestión de Requisitos}
La gestión de requisitos es el proceso sistemático de identificar, documentar, organizar, rastrear y controlar los cambios en los requisitos de un sistema de software durante todo su ciclo de vida de desarrollo.

\subsubsection{Trazabilidad}
Capacidad de seguir la vida de un requisito tanto hacia adelante como hacia atrás (desde sus orígenes, a través de su desarrollo y especificación, hasta su implementación y uso).

\subsubsection{Versionado}
Control sistemático de las versiones y revisiones de los requisitos, permitiendo rastrear cambios y mantener un historial completo de modificaciones.

\subsubsection{Matriz de Trazabilidad}
Documento que vincula los requisitos a lo largo de todo el proceso de validación, desde la definición inicial hasta las pruebas finales del sistema.

\subsection{Criterios de Evaluación}

Para la selección de la herramienta más adecuada se establecieron los siguientes criterios de evaluación:

\begin{itemize}
    \item \textbf{Funcionalidad}: Capacidades específicas para gestión de requisitos
    \item \textbf{Usabilidad}: Facilidad de uso e interfaz intuitiva
    \item \textbf{Costo}: Accesibilidad económica para proyectos académicos
    \item \textbf{Colaboración}: Soporte para trabajo en equipo
    \item \textbf{Integración}: Compatibilidad con otras herramientas
    \item \textbf{Escalabilidad}: Capacidad de crecer con el proyecto
    \item \textbf{Soporte}: Disponibilidad de documentación y ayuda
\end{itemize}

\newpage

\subsection{Análisis Comparativo de Herramientas}

\subsubsection{Cuadro Comparativo General}

\begin{longtable}{|p{2.5cm}|p{2cm}|p{2cm}|p{2cm}|p{2cm}|p{2.5cm}|}
\hline
\rowcolor{headercolor}
\textcolor{white}{\textbf{Herramienta}} & 
\textcolor{white}{\textbf{Tipo}} & 
\textcolor{white}{\textbf{Costo}} & 
\textcolor{white}{\textbf{Facilidad de Uso}} & 
\textcolor{white}{\textbf{Colaboración}} & 
\textcolor{white}{\textbf{Ideal para}} \\
\hline

\textbf{Jira} & 
Comercial & 
Freemium & 
Media & 
Excelente & 
Proyectos ágiles \\
\hline
\rowcolor{lightgray}

\textbf{ReqView} & 
Comercial & 
Pago & 
Alta & 
Buena & 
Requisitos formales \\
\hline

\textbf{Trello} & 
Freemium & 
Gratuito/Pago & 
Muy Alta & 
Excelente & 
Proyectos simples \\
\hline
\rowcolor{lightgray}

\textbf{ClickUp} & 
Freemium & 
Gratuito/Pago & 
Alta & 
Excelente & 
Gestión integral \\
\hline

\textbf{GitHub Projects} & 
Gratuito & 
Totalmente Gratuito & 
Media & 
Excelente & 
Proyectos de código \\
\hline
\rowcolor{lightgray}

\textbf{Azure DevOps} & 
Comercial & 
Freemium & 
Media & 
Excelente & 
Ecosistema Microsoft \\
\hline

\textbf{Notion} & 
Freemium & 
Gratuito/Pago & 
Alta & 
Buena & 
Documentación \\
\hline

\end{longtable}

\subsubsection{Análisis Detallado por Funcionalidades}

\begin{longtable}{|p{2.5cm}|p{1.5cm}|p{1.5cm}|p{1.5cm}|p{1.5cm}|p{1.5cm}|p{1.5cm}|}
\hline
\rowcolor{headercolor}
\textcolor{white}{\textbf{Herramienta}} & 
\textcolor{white}{\textbf{Trazabilidad}} & 
\textcolor{white}{\textbf{Versionado}} & 
\textcolor{white}{\textbf{Priorización}} & 
\textcolor{white}{\textbf{Reportes}} & 
\textcolor{white}{\textbf{Integración}} & 
\textcolor{white}{\textbf{Puntuación}} \\
\hline

\textbf{Jira} & 
Excelente & 
Excelente & 
Excelente & 
Excelente & 
Excelente & 
9.5/10 \\
\hline
\rowcolor{lightgray}

\textbf{ReqView} & 
Excelente & 
Excelente & 
Buena & 
Excelente & 
Buena & 
8.5/10 \\
\hline

\textbf{Trello} & 
Básica & 
Básica & 
Buena & 
Básica & 
Buena & 
6.0/10 \\
\hline
\rowcolor{lightgray}

\textbf{ClickUp} & 
Buena & 
Buena & 
Excelente & 
Buena & 
Excelente & 
8.0/10 \\
\hline

\textbf{GitHub Projects} & 
Buena & 
Buena & 
Buena & 
Buena & 
Excelente & 
7.8/10 \\
\hline
\rowcolor{lightgray}

\textbf{Azure DevOps} & 
Excelente & 
Excelente & 
Excelente & 
Excelente & 
Excelente & 
9.2/10 \\
\hline

\textbf{Notion} & 
Básica & 
Básica & 
Básica & 
Básica & 
Buena & 
5.5/10 \\
\hline

\end{longtable}

\subsection{Descripción Detallada de Herramientas Principales}

\subsubsection{Jira (Atlassian)}

\textbf{Descripción:}
Jira es una herramienta de gestión de proyectos y seguimiento de issues ampliamente utilizada en el desarrollo de software. Originalmente diseñada para bug tracking, ha evolucionado para soportar metodologías ágiles y gestión completa de requisitos.

\textbf{Características principales:}
\begin{itemize}
    \item Sistema robusto de tracking de issues y requisitos
    \item Flujos de trabajo personalizables (workflows)
    \item Campos personalizados para atributos de requisitos
    \item Matriz de trazabilidad mediante enlaces entre issues
    \item Integración nativa con herramientas de desarrollo
    \item Dashboards y reportes avanzados
    \item Control de versiones integrado
    \item Gestión de cambios estructurada
\end{itemize}

\textbf{Ventajas:}
\begin{itemize}
    \item Amplia funcionalidad para gestión de requisitos
    \item Ecosistema robusto de plugins y extensiones
    \item Excelente para metodologías ágiles
    \item Integración con herramientas de desarrollo populares
    \item Versión gratuita para equipos pequeños
\end{itemize}

\textbf{Desventajas:}
\begin{itemize}
    \item Curva de aprendizaje inicial
    \item Puede ser sobrecargado para proyectos simples
    \item Costo elevado para equipos grandes
\end{itemize}

\subsubsection{ReqView}

\textbf{Descripción:}
ReqView es una herramienta especializada en gestión de requisitos que se enfoca en proporcionar funcionalidades específicas para documentación formal de requisitos y cumplimiento de estándares como ISO 26262 y DO-178C.

\textbf{Características principales:}
\begin{itemize}
    \item Editor especializado para documentos de requisitos
    \item Matriz de trazabilidad automática
    \item Atributos personalizables para requisitos
    \item Versionado y control de cambios
    \item Importación/exportación de documentos
    \item Colaboración en tiempo real
    \item Validación de requisitos
\end{itemize}

\textbf{Ventajas:}
\begin{itemize}
    \item Especializada específicamente en requisitos
    \item Interfaz intuitiva para documentación
    \item Excelente trazabilidad
    \item Cumplimiento de estándares industriales
\end{itemize}

\textbf{Desventajas:}
\begin{itemize}
    \item Costo elevado
    \item Funcionalidad limitada fuera de gestión de requisitos
    \item Menor ecosistema de integraciones
\end{itemize}

\subsubsection{GitHub Projects}

\textbf{Descripción:}
GitHub Projects es la herramienta nativa de gestión de proyectos de GitHub, completamente gratuita para proyectos públicos y privados. Está integrada directamente con los repositorios de código y proporciona funcionalidades robustas para la gestión de requisitos en proyectos de desarrollo de software.

\textbf{Características principales:}
\begin{itemize}
    \item Tableros Kanban personalizables
    \item Issues integrados para requisitos
    \item Campos personalizados ilimitados
    \item Automatización de flujos de trabajo
    \item Integración nativa con Git y código
    \item Seguimiento de milestone y releases
    \item Colaboración en tiempo real
    \item Vistas múltiples (tablero, tabla, roadmap)
\end{itemize}

\textbf{Ventajas:}
\begin{itemize}
    \item Completamente gratuito para uso académico y profesional
    \item Integración perfecta con desarrollo de código
    \item Interface familiar para desarrolladores
    \item Excelente para metodologías ágiles
    \item Comunidad activa y soporte extenso
    \item Sin límites de usuarios o proyectos
\end{itemize}

\textbf{Desventajas:}
\begin{itemize}
    \item Requiere familiarización con ecosistema GitHub
    \item Menos funciones específicas de requisitos formales
    \item Dependiente de conexión a internet
\end{itemize}

\subsubsection{ClickUp}

\textbf{Descripción:}
ClickUp es una plataforma de productividad todo-en-uno que incluye capacidades robustas de gestión de requisitos dentro de un entorno más amplio de gestión de proyectos. Ofrece un plan gratuito generoso para equipos pequeños.

\textbf{Características principales:}
\begin{itemize}
    \item Múltiples vistas (lista, tablero, Gantt, calendario)
    \item Campos personalizados (limitados en plan gratuito)
    \item Automatización básica de flujos de trabajo
    \item Seguimiento de tiempo integrado
    \item Colaboración en tiempo real
    \item Dashboards básicos
    \item Gestión de documentos integrada
\end{itemize}

\textbf{Ventajas:}
\begin{itemize}
    \item Plan gratuito generoso para equipos pequeños
    \item Interfaz moderna e intuitiva
    \item Versátil y completo
    \item Excelente para equipos diversos
\end{itemize}

\textbf{Desventajas:}
\begin{itemize}
    \item Limitaciones significativas en plan gratuito
    \item Funciones avanzadas requieren planes pagos
    \item Puede ser abrumador inicialmente
\end{itemize}

\subsection{Matriz de Evaluación Ponderada}

\begin{table}[H]
\centering
\begin{tabular}{|p{2.5cm}|c|c|c|c|c|c|}
\hline
\rowcolor{headercolor}
\textcolor{white}{\textbf{Criterio}} & 
\textcolor{white}{\textbf{Peso}} & 
\textcolor{white}{\textbf{GitHub Projects}} & 
\textcolor{white}{\textbf{ClickUp}} & 
\textcolor{white}{\textbf{Trello}} & 
\textcolor{white}{\textbf{Notion}} & 
\textcolor{white}{\textbf{Jira}} \\
\hline

Funcionalidad & 25\% & 8 & 8 & 6 & 6 & 9 \\
\hline
\rowcolor{lightgray}

Facilidad de Uso & 20\% & 7 & 9 & 10 & 9 & 7 \\
\hline

Costo (Gratuito) & 30\% & 10 & 8 & 10 & 9 & 7 \\
\hline
\rowcolor{lightgray}

Colaboración & 15\% & 9 & 9 & 9 & 7 & 9 \\
\hline

Integración & 10\% & 10 & 8 & 7 & 6 & 10 \\
\hline

\textbf{Total Ponderado} & & \textbf{8.7} & \textbf{8.3} & \textbf{8.5} & \textbf{7.3} & \textbf{8.0} \\
\hline

\end{tabular}
\caption{Matriz de Evaluación Ponderada de Herramientas}
\end{table}

\subsection{Herramienta Seleccionada: GitHub Projects}

\subsubsection{Justificación de la Elección}

Después del análisis comparativo realizado con énfasis en herramientas gratuitas para proyectos académicos, se selecciona \textbf{GitHub Projects} como la herramienta principal para la gestión de requisitos del proyecto TutoESPEcialistas por las siguientes razones:

\begin{enumerate}
    \item \textbf{Puntuación más alta en la matriz ponderada (8.7/10)}
    \item \textbf{Completamente gratuito}: Sin restricciones para proyectos académicos
    \item \textbf{Integración perfecta}: Nativo con desarrollo de código en GitHub
    \item \textbf{Sin límites}: No hay restricciones de usuarios, proyectos o funcionalidades
    \item \textbf{Escalabilidad}: Puede crecer con el proyecto sin costos adicionales
    \item \textbf{Familiaridad}: Ampliamente usado en entornos académicos y profesionales
    \item \textbf{Documentación extensiva}: Excelente soporte y tutoriales gratuitos
\end{enumerate}

\subsubsection{Funciones Específicas Útiles para el PGR}

\textbf{1. Gestión de Estados}
\begin{itemize}
    \item Estados personalizables mediante labels (Propuesto, En Análisis, Aprobado, Implementado, Verificado)
    \item Flujos de trabajo automatizados con GitHub Actions
    \item Seguimiento visual del progreso mediante project boards
    \item Integración con pull requests y commits
\end{itemize}

\textbf{2. Trazabilidad}
\begin{itemize}
    \item Referencias entre issues mediante \# y enlaces
    \item Milestone tracking para agrupación de requisitos
    \item Vinculación automática con commits y pull requests
    \item Búsqueda avanzada para rastrear requisitos relacionados
    \item Historial completo de actividad
\end{itemize}

\textbf{3. Versionado}
\begin{itemize}
    \item Historial completo de cambios en cada issue
    \item Comentarios y notas de versión integrados
    \item Integración nativa con Git para versionado de código
    \item Timeline de actividades detallado
\end{itemize}

\textbf{4. Priorización}
\begin{itemize}
    \item Labels personalizables para prioridad
    \item Ordenamiento manual en project boards
    \item Filtros avanzados por prioridad
    \item Milestones para planificación temporal
\end{itemize}

\textbf{5. Visualización}
\begin{itemize}
    \item Múltiples vistas: Board, Table, Roadmap
    \item Gráficos de progreso integrados
    \item Dashboard personalizable del proyecto
    \item Integración con insights y analytics
\end{itemize}

\subsection{Implementación de GitHub Projects para TutoESPEcialistas}

\subsubsection{Configuración del Repositorio y Proyecto}

\textbf{Estructura Propuesta:}
\begin{itemize}
    \item \textbf{Repositorio}: "TutoESPEcialistas" (público para fines académicos)
    \item \textbf{Project}: "Gestión de Requisitos TutoESPEcialistas"
    \item \textbf{Milestones}: 
        \begin{itemize}
            \item Análisis de Requisitos (v1.0)
            \item Diseño del Sistema (v2.0)
            \item Implementación Fase 1 (v3.0)
            \item Pruebas y Validación (v4.0)
        \end{itemize}
    \item \textbf{Labels}: Por tipo, prioridad, estado y módulo del sistema
\end{itemize}

\subsubsection{Configuración de Labels y Templates}

Para cada requisito se configurarán los siguientes labels:

\begin{table}[H]
\centering
\begin{tabular}{|p{3cm}|p{2cm}|p{6cm}|}
\hline
\rowcolor{headercolor}
\textcolor{white}{\textbf{Categoría}} & 
\textcolor{white}{\textbf{Label}} & 
\textcolor{white}{\textbf{Descripción}} \\
\hline

Tipo & req-funcional & Requisito funcional \\
\hline
\rowcolor{lightgray}

Tipo & req-no-funcional & Requisito no funcional \\
\hline

Tipo & restriccion & Restricción del sistema \\
\hline
\rowcolor{lightgray}

Prioridad & prioridad-alta & Prioridad alta \\
\hline

Prioridad & prioridad-media & Prioridad media \\
\hline
\rowcolor{lightgray}

Prioridad & prioridad-baja & Prioridad baja \\
\hline

Estado & identificado & Requisito documentado \\
\hline
\rowcolor{lightgray}

Estado & en-análisis & En proceso de revisión \\
\hline

Estado & aprobado & Validado por stakeholders \\
\hline
\rowcolor{lightgray}

Módulo & autenticacion & Módulo de autenticación \\
\hline

Módulo & tutorias & Gestión de tutorías \\
\hline
\rowcolor{lightgray}

Módulo & evaluacion & Sistema de evaluación \\
\hline

\end{tabular}
\caption{Configuración de Labels en GitHub}
\end{table}

\subsubsection{Template de Issues para Requisitos}

Se creará un template estándar para documentar requisitos como issues:

\begin{verbatim}
## RF/RNF-XXX: [Nombre del Requisito]

### Información Básica
- **ID**: RF/RNF-XXX
- **Versión**: X.X
- **Autor**: @username
- **Fuente**: [Entrevista/Documento/etc.]

### Descripción
[Descripción detallada del requisito]

### Criterios de Aceptación
- [ ] Criterio 1
- [ ] Criterio 2
- [ ] Criterio 3

### Referencias Cruzadas
- Relacionado con: #issue-number
- Depende de: #issue-number
- Bloquea: #issue-number

### Atributos
- **Estabilidad**: [Fijo/Establecido/Volátil]
- **Riesgo**: [Alto/Medio/Bajo]
- **Prioridad**: [Alta/Media/Baja]

### Trazabilidad
[Documento o stakeholder origen]
\end{verbatim}

\subsection{Ejemplo de Uso Práctico}

\subsubsection{Creación de un Requisito}

\textbf{Paso 1: Crear nuevo issue}
\begin{itemize}
    \item Acceder al repositorio "TutoESPEcialistas"
    \item Crear nuevo issue usando el template de requisitos
    \item Título: "RF02 - Solicitud de tutorías"
    \item Asignar al analista responsable
\end{itemize}

\textbf{Paso 2: Completar información del requisito}
\begin{itemize}
    \item ID Requisito: RF02
    \item Versión: 1.5
    \item Autor: @dennison-chalacan
    \item Fuente: Entrevistas y encuestas
    \item Estabilidad: Establecido
    \item Riesgo: Alto
    \item Prioridad: Alta
\end{itemize}

\textbf{Paso 3: Establecer labels y referencias}
\begin{itemize}
    \item Aplicar labels: req-funcional, prioridad-alta, modulo-tutorias
    \item Vincular con issues relacionados usando \#1 (RF01), \#3 (RF03)
    \item Asignar al milestone correspondiente
    \item Agregar al project board
\end{itemize}

\textbf{Paso 4: Documentar y seguimiento}
\begin{itemize}
    \item Completar descripción detallada en el issue
    \item Definir criterios de aceptación
    \item Agregar comentarios del proceso de análisis
    \item Mover a columna correspondiente en project board
\end{itemize}

\subsubsection{Seguimiento y Control}

\textbf{Dashboard de Métricas:}
\begin{itemize}
    \item Número total de requisitos por estado
    \item Distribución por prioridad
    \item Progreso de implementación
    \item Requisitos con cambios recientes
    \item Tiempo promedio por estado
\end{itemize}

\textbf{Reportes Automáticos:}
\begin{itemize}
    \item Reporte semanal de estado de requisitos
    \item Lista de cambios pendientes de aprobación
    \item Matriz de trazabilidad actualizada
    \item Requisitos en riesgo por tiempo de permanencia en estado
\end{itemize}

\subsection{Ventajas de ClickUp para TutoESPEcialistas}

\begin{enumerate}
    \item \textbf{Accesibilidad}: Plan gratuito suficiente para el alcance del proyecto
    \item \textbf{Flexibilidad}: Adaptable a las necesidades específicas del proyecto académico
    \item \textbf{Colaboración}: Facilita el trabajo conjunto del equipo de desarrollo
    \item \textbf{Escalabilidad}: Puede crecer con el proyecto si se expande en el futuro
    \item \textbf{Integración}: Compatible con herramientas de desarrollo comunes
    \item \textbf{Documentación}: Excelente sistema de ayuda y tutoriales
    \item \textbf{Visualización}: Múltiples formas de ver y analizar los requisitos
\end{enumerate}

\subsection{Limitaciones y Consideraciones}

\begin{enumerate}
    \item \textbf{Funcionalidades avanzadas}: Algunas características específicas de gestión de requisitos pueden requerir configuración adicional
    \item \textbf{Curva de aprendizaje}: Aunque intuitivo, requiere tiempo para dominar todas las características
    \item \textbf{Dependencia de internet}: Herramienta basada en la nube que requiere conexión constante
    \item \textbf{Límites del plan gratuito}: Algunas funciones avanzadas solo están disponibles en planes pagos
\end{enumerate}

\subsection{Plan de Implementación}

\subsubsection{Fase 1: Configuración Inicial (Semana 1)}
\begin{itemize}
    \item Creación del workspace y estructura de folders
    \item Configuración de campos personalizados
    \item Definición de estados y flujos de trabajo
    \item Invitación y configuración de permisos del equipo
\end{itemize}

\subsubsection{Fase 2: Migración de Requisitos (Semana 2)}
\begin{itemize}
    \item Importación de requisitos existentes
    \item Establecimiento de relaciones y dependencias
    \item Configuración de dashboards y vistas
    \item Validación de la configuración con el equipo
\end{itemize}

\subsubsection{Fase 3: Capacitación y Adopción (Semana 3)}
\begin{itemize}
    \item Capacitación del equipo en el uso de la herramienta
    \item Establecimiento de procesos y procedimientos
    \item Pruebas de los flujos de trabajo configurados
    \item Ajustes finales basados en retroalimentación
\end{itemize}

\subsubsection{Fase 4: Operación y Mantenimiento (Ongoing)}
\begin{itemize}
    \item Uso diario de la herramienta para gestión de requisitos
    \item Monitoreo de métricas y generación de reportes
    \item Optimización continua de procesos
    \item Evaluación periódica de efectividad
\end{itemize}

\section{Conclusiones}

La selección de GitHub Projects como herramienta de soporte para la gestión de requisitos del proyecto TutoESPEcialistas se basa en un análisis objetivo que prioriza las necesidades específicas de un proyecto académico, donde el costo es un factor determinante.

GitHub Projects ofrece un balance ideal entre funcionalidad completa, accesibilidad económica (completamente gratuito) y facilidad de adopción para equipos académicos. Su integración nativa con el desarrollo de software lo convierte en la elección natural para proyectos que combinarán gestión de requisitos con desarrollo de código.

La implementación gradual propuesta asegura una adopción exitosa de la herramienta, aprovechando al máximo sus capacidades sin restricciones económicas, lo que es fundamental para proyectos estudiantiles y académicos.

\section{Referencias}

\begin{itemize}
    \item GitHub. (2025). \textit{GitHub Projects Documentation}. Recuperado de https://docs.github.com/en/issues/planning-and-tracking-with-projects
    \item GitHub. (2025). \textit{GitHub Issues Documentation}. Recuperado de https://docs.github.com/en/issues
    \item Atlassian. (2025). \textit{Jira Software Documentation}. Recuperado de https://www.atlassian.com/software/jira
    \item Trello. (2025). \textit{Trello Features and Documentation}. Recuperado de https://trello.com
    \item IEEE. (1998). \textit{IEEE Recommended Practice for Software Requirements Specifications - IEEE Std 830-1998}
    \item Sommerville, I. (2016). \textit{Software Engineering} (10th ed.). Pearson Education Limited.
    \item Pressman, R. S. (2014). \textit{Software Engineering: A Practitioner's Approach} (8th ed.). McGraw-Hill Education.
\end{itemize}

\end{document}
